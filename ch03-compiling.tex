\chapter{Compiling exceptions}

The basic approach to compiling exceptions was shown in \cite{gmh:exceptions}.
This article provides excellent insight and inspiration how such code can be written
in a language such as Haskell.
However, peculiarities of programming with dependent types will make us
diverge a bit, in details at first, in parts of the design later.

\section{A simple exceptionless language}

All high-level languages in this thesis will be languages of simple, typed
expressions.

Our first language will feature only natural numbers and addition, not even
exceptions. We will implement it to build the auxiliary ecosystem of the
compiler and the skeleton of our development.

\subsection{Type universe}

The first thing we will introduce is the type universe representing the set of
types of expressions of the high-level language. We will index our types with
values of this type (most notably the type \ident{Exp u} of expressions, to
indicate what the type of the expression is), and while we could use Agda types
directly for indexing, with an explicit universe, we get decidable equality of
types and all datatypes conveniently in \ident{Set}. We could parametrize our
modules with the universe but let us just define a fixed one for the sake of
simplicity.

\begin{code}
  infixr 5 _=>_
  data U : Set where
    nat : U
    _=>_ : U -> U -> U
\end{code}

\noindent We will also need the interpretation function \ident{el} that maps
types of our simple language to Agda types so that we can use Agda values in
the modelled language, talk about its denotational semantics etc.

\begin{code}
  el : U -> Set
  el nat = Nat
  el (s => t) = el s -> el t
\end{code}

\noindent And, as promised, equality is decidable on the elements of the
universe.

\begin{code}
  -- open import Relation.Nullary using (Dec)
  data Dec (p : Set) : Set where
    yes : p -> Dec p
    no  : ~ p -> Dec p

  -- open import Relation.Binary.PropositionalEquality using (_==_)
  data _==_ {a : Set} (x : a) : a -> Set where
    refl : x == x

  eqDecU : (u v : U) -> Dec (u == v)
  -- body omitted
\end{code}

\noindent We omit the body of \ident{eqDecU} because it's just an ordinary
uninteresting case analysis.

\subsection{Expressions}

The core of the language we are going to model consists of its expressions, of
course. For now, we will support nothing more than (numeric) literals and addition.
However, for further extensibility, we separate the type of binary operators.

The type of binary operators is indexed by the types of the two values that
the operator accepts as arguments; the third index represents the type of
the result of application of the operator on the two values.

\begin{code}
  data Op : U -> U -> U -> Set where
    Plus : Op nat nat nat
\end{code}

\noindent Now we can define the expressions: literals and binary operators.
Note that we index the datatype with elements of the universe \ident{U} so that
the Agda type of an expression also incorporates the type of the expression in
the modelled language.

\begin{code}
  data Exp : U -> Set where
    -- Literals
    Lit : forall {u} -> el u -> Exp u
    -- Binary operators
    Bin : forall {u v w} -> Bin u v w -> Exp u -> Exp v -> Exp w
\end{code}

\subsection{Semantics of expressions}

Our definition of the high-level language would not be complete without giving
the denotational semantics of its expressions. This is done in the following
pair of simple functions.

\begin{code}
  denOp : forall {u v w} -> Op u v w -> el u -> el v -> el w
  denOp Plus = _+_

  denExp : forall {u} -> Exp u -> el u
  denExp (Lit x) = x
  denExp (Bin op l r) = denOp op (denExp l) (denExp r)
\end{code}

\noindent The separate type of binary operators deserves a separate function
converting an operator to a binary Agda function of the appropriate type.

Expressions are then turned into Agda values recursively; literals in a trivial
way, binary-operator expressions using the denotation of the corresponding
operator.

\subsection{Virtual machine}

We will use a very simple stack machine to run the compiled code.

\subsubsection{Stack}

The stack of
the machine is just a cons-list of values, indexed by types (elements of
\ident{U}) of the values pushed on the stack.  This means that just by looking
at the type of the stack, we can tell how many elements it contains and what
types they have.
\begin{code}
  -- open import Data.List
  infixr 5 _::_
  data List (a : Set) : Set where
    [] : List a
    _::_ : a -> List a -> List a

  Shape : Set
  Shape = List U

  infixr 5 _\scons_
  data Stack : Shape -> Set where
    snil : Stack []
    _\scons_ : forall {u s} -> el u -> Stack s -> Stack (u :: s)
\end{code}
\noindent The literal \ident{snil} represents the empty stack; new values are
pushed onto it using the infix constructor \ident{\_\scons\_}.

\subsubsection{Instructions}

At this stage, the machine supports only two instructions: \ident{PUSH}
and \ident{ADD}.
\begin{code}
  data Instr : Shape -> Shape -> Set where
    PUSH : forall {u s} -> el u -> Instr s (u :: s)
    ADD : forall s -> Instr (nat :: nat :: s) (nat :: s)
\end{code}
The type of instructions is indexed by their action on stack. The first shape
argument is the required stack shape so that the instruction can be executed;
the second shape argument is the resulting shape of the stack after the
instruction has been executed.

For example, the instruction \ident{PUSH} takes any value of the type \ident{el u}
and pushes it onto a stack having any shape \ident{s}, creating a new
stack of the shape \ident{u :: s}.

The instruction \ident{ADD} represents popping two natural numbers from the
stack of any shape with two \ident{nat}s on top of it, (hence
\ident{nat :: nat :: s})
and subsequently pushing their sum onto it, resulting in the shape
\ident{nat :: s}.

\subsubsection{Code}

Finally, code for the stack machine is a sequence of instructions, where
type indices of subsequent instructions match. For example, if one instruction
in the sequence produces a stack of the shape \ident{nat :: nat :: s},
we want the next instruction in the code sequence to accept this shape.

If we regard
$\midt{Instr} : \midt{Shape} \to \midt{Shape} \to \midt{Set}$
as a binary relation on \ident{Shape}, then code is the \emph{transitive reflexive closure}
of \ident{Instr}, which is already included in the Agda standard library as the
module \ident{Data.Star}.

\begin{code}
  -- require import Data.Star
  infixr 5 _<|_
  data Star {a b : Set} (R : a -> b -> Set) : a -> b -> Set where
    \nil : forall {x} -> Star R x x
    _<|_ : forall {x y z} -> R x y -> Star R y z -> Star R x z

  Code : Shape -> Shape -> Set
  Code = Star Instr
\end{code}

\noindent The type of instruction sequences is indexed in exactly the same
manner as the type of single instructions: the first index represents the
acceptable shape of stack before execution of the piece of code; the second
index represents the shape of stack after its execution.

Let us conclude this section with an utility function for concatenation of
instruction sequences, which is actually also included in \ident{Data.Star}.

\begin{code}
  infixr 5 _\app_
  _\app_ : forall {R x y z} -> Star R x y -> Star R y z -> Star R x z
  _\app_ \nil ys = ys
  _\app_ (x <| xs) ys = x <| xs \app ys
\end{code}

\subsection{Execution}

Now we will describe how the machine executes instructions, that is,
the operational semantics of the low-level language.

At this stage, the state of the machine is fully described by just its stack. This
means that there are no other state variables, registers or any additional
memory.

\subsubsection{Instructions}

Let us describe the effects of single instructions on the state of the machine
(that is, on the stack).

\begin{code}
  execInstr : forall {s t} -> Instr s t -> Stack s -> Stack t
  execInstr (PUSH x) st = x \scons st
  execInstr ADD (x \scons y \scons st) = (x + y) \scons st
\end{code}

\noindent The above function simply says that
\begin{itemize}
  \item the effect of the instruction \ident{PUSH} is pushing the attached
    value onto the stack. This consistently extends the information contained
    in the type of \ident{PUSH x}.\footnote{\ident{Instr s (u $::$ s)} -- for
    some \ident{u} and \ident{s}. This type is interpreted as ,,\ident{PUSH x}
    pushes some value of type \ident{u} onto the stack''.}
    What the type does not say (and \ident{execInstr}
    does) is what value exactly this is.
  \item the effect of the instruction \ident{ADD} pops two \ident{nat}s from
    the top of the stack and pushes their sum back.
\end{itemize}

Note that in this definition, we already reap some benefits of dependently
typed programming.

First, of course, Agda checks types of the terms behind
the scenes and the machinery of types we have designed so far ensures that
in the case for \ident{PUSH x}, pushing the value \ident{x} always yields
a stack of the desired shape.

Second, in the case for \ident{ADD}, the types ensure that there are always two
\ident{nat}s on top of the stack and we can safely pattern-match with the
pattern \ident{x} \scons \ident{y} \scons \ident{st} -- because this match
will always succeed (and no other patterns for the \ident{ADD} case are needed). 

Thus the above definition complies to the type signatures involved (relatively
solid hints of correctness) and it is \emph{total} (esp. no pattern match failures),
while compilers of non-dependently typed languages, like OCaml or Haskell,
would complain about non-exhaustive patterns here --- there is no way to tell them
that, for example, we needn't deal with empty stacks when executing \ident{ADD}.
% TODO: ktory paper tvrdi, ze to je nevyhnutne?

\subsubsection{Code}

Execution of code is then just a left fold over the sequence of instructions,
accepting the initial and yielding the resulting state of the machine.

\begin{code}
  execCode : forall {s t} -> Code s t -> Stack s -> Stack t
  execCode \nil st = st
  execCode (i <| is) st = execCode is (execInstr i st)
\end{code}

\noindent Execution of empty code has no effect on the stack; if the code
contains instructions, then the first instruction is executed and on the
resulting stack, the rest of code is executed.

\subsection{Compiler}

Compiling our simple high-level language for a stack machine is easy. The
central idea is that execution of an expression of some type is equivalent to
pushing its value onto the stack. Literal values are then pushed on the stack
directly; binary-operator expressions first evaluate both operands, effectively
putting their values on the top of the stack, and then execute the appropriate
instruction, determined by the operator. This instruction pops the top
two values from the stack as its operands and pushes the result back.

\begin{code}
  -- Syntactic sugar, promote an Instr to singleton Code
  [[_\;]] : forall {s t} -> Instr s t -> Code s t
  [[_\;]] i = i <| \nil

  -- Determine what instruction performs the required calculation
  opInstr : forall {u v w} -> Op u v w -> forall {s} -> Instr (u :: v :: s) (w :: s)
  opInstr Plus = ADD

  -- Turn the expression into code
  compile : forall {u} -> Exp u -> forall {s} -> Code s (u :: s)
  compile (Lit x) = [[ PUSH x ]]
  compile (Bin op l r) = compile r \app compile r \app [[ opInstr op ]]
\end{code}

Again, behind the scenes, Agda ensures that all types match and the code
compiled by this function will not make the stack machine fail.\footnote{
To be fair, this is already a property of \ident{Code},
,,inherited'' by the function \ident{compile} via its return type.
However, it does constrain possible definitions of the function \ident{compile}.}
For example,
there is no way to have the function \ident{compile} output code where
\ident{ADD} would not have two \ident{nat}s on the top of the
stack.

\subsection{Correctness}

% TODO?

\section{Adding exceptions, GMH-style}

\section{Termination}
