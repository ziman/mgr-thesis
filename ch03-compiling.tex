\chapter{Compiling exceptions totally correctly}

\todo{Blah blah about differences between FP and total FP, substantiate the
title of the section}

The basic approach to compiling exceptions was shown in \cite{gmh:exceptions}.
This article provides excellent insight and inspiration how such code can be written
in a language such as Haskell.
However, peculiarities of programming with dependent types will make us
diverge a bit, in details at first, in parts of the design later.

\section{A simple exceptionless language}

All high-level languages in this thesis will be languages of simple, typed
expressions.

Our first language will feature only natural numbers and addition, not even
exceptions. We will implement it to build the auxiliary ecosystem of the
compiler and the skeleton of our development.

% branch: no-exceptions

\subsection{Type universe}

The first thing we will introduce is the type universe representing the set of
types of expressions of the high-level language. We will index our types with
values of this type (most notably the type \ident{Exp u} of expressions, to
indicate what the type of the expression is), and while we could use Agda types
directly for indexing, with an explicit universe, we get decidable equality of
types and all datatypes conveniently in \ident{Set}. We could parametrize our
modules with the universe but let us just define a fixed one for the sake of
simplicity.

\begin{code}
  infixr 5 _=>\_
  data U : Set where
    nat : U
    _=>\_ : U -> U -> U
\end{code}

\todo{Omit the unused $\Rightarrow$ constructor entirely?}

\noindent We will also need the interpretation function \ident{el} that maps
types of our simple language to Agda types so that we can use Agda values in
the modelled language, talk about its denotational semantics etc.

\begin{code}
  el : U -> Set
  el nat = Nat
  el (s => t) = el s -> el t
\end{code}

\noindent And, as promised, equality is decidable on the elements of the
universe.

\begin{code}
  -- open import Relation.Nullary using (Dec)
  data Dec (p : Set) : Set where
    yes : p -> Dec p
    no  : ~ p -> Dec p

  -- open import Relation.Binary.PropositionalEquality using (_==\_)
  data _==\_ {a : Set} (x : a) : a -> Set where
    refl : x == x

  eqDecU : (u v : U) -> Dec (u == v)
  -- body omitted
\end{code}

\noindent We omit the body of \ident{eqDecU} because it's just an ordinary
uninteresting case analysis.

In the following text, we will be a bit lax on wording related to type
universes.  We will be talking about ``expressions of the type \ident{u}'',
while actually referring to ``terms that represent expressions of \emph{the
type denoted by \ident{u}}''. This simplified approach can hardly cause
confusion, while adhering to precise wording in every situation at all costs
would lead to incomprehensible sentences.

\subsection{Expressions}

The core of the language we are going to model consists of its expressions, of
course. For now, we will support nothing more than (numeric) literals and addition.
However, for further extensibility, we separate the type of binary operators.

The type of binary operators is indexed by the types of the two values that
the operator accepts as arguments; the third index represents the type of
the result of application of the operator on the two values.

\begin{code}
  data Op : U -> U -> U -> Set where
    Plus : Op nat nat nat
\end{code}

\noindent Now we can define the expressions: literals and binary operators.
Note that we index the datatype with elements of the universe \ident{U} so that
the Agda type of an expression also incorporates the type of the expression in
the modelled language.

\begin{code}
  data Exp : U -> Set where
    -- Literals
    Lit : forall {u} -> el u -> Exp u
    -- Binary operators
    Bin : forall {u v w} -> Bin u v w -> Exp u -> Exp v -> Exp w
\end{code}

\subsection{Semantics of expressions}

Our definition of the high-level language would not be complete without giving
the denotational semantics of its expressions. This is done in the following
pair of simple functions.

\begin{code}
  denOp : forall {u v w} -> Op u v w -> el u -> el v -> el w
  denOp Plus = _+\_

  denExp : forall {u} -> Exp u -> el u
  denExp (Lit x) = x
  denExp (Bin op l r) = denOp op (denExp l) (denExp r)
\end{code}

\noindent The separate type of binary operators deserves a separate function
converting an operator to a binary Agda function of the appropriate type.

Expressions are then turned into Agda values recursively; literals in a trivial
way, binary-operator expressions using the denotation of the corresponding
operator.

\subsection{Virtual machine}

We will use a very simple stack machine to run the compiled code.

\subsubsection{Stack}

The stack of
the machine is just a cons-list of values, indexed by types (elements of
\ident{U}) of the values pushed on the stack.  This means that just by looking
at the type of the stack, we can tell how many elements it contains and what
types they have.
\begin{code}
  -- open import Data.List
  infixr 5 _::_
  data List (a : Set) : Set where
    [] : List a
    _::_ : a -> List a -> List a

  Shape : Set
  Shape = List U

  infixr 5 _\scons\_
  data Stack : Shape -> Set where
    snil : Stack []
    _\scons\_ : forall {u s} -> el u -> Stack s -> Stack (u :: s)
\end{code}
\noindent The literal \ident{snil} represents the empty stack; new values are
pushed onto it using the infix constructor \ident{\bin{\scons}}.

\subsubsection{Instructions}

At this stage, the machine supports only two instructions: \ident{PUSH}
and \ident{ADD}.
\begin{code}
  data Instr : Shape -> Shape -> Set where
    PUSH : forall {u s} -> el u -> Instr s (u :: s)
    ADD : forall s -> Instr (nat :: nat :: s) (nat :: s)
\end{code}
The type of instructions is indexed by their action on stack. The first shape
argument is the required stack shape so that the instruction can be executed;
the second shape argument is the resulting shape of the stack after the
instruction has been executed.

For example, the instruction \ident{PUSH} takes any value of the type \ident{el u}
and pushes it onto a stack having any shape \ident{s}, creating a new
stack of the shape \ident{u :: s}.

The instruction \ident{ADD} represents popping two natural numbers from the
stack of any shape with two \ident{nat}s on top of it, (hence
\ident{nat :: nat :: s})
and subsequently pushing their sum onto it, resulting in the shape
\ident{nat :: s}.

\subsubsection{Code}

Finally, code for the stack machine is a sequence of instructions, where
type indices of subsequent instructions match. For example, if one instruction
in the sequence produces a stack of the shape \ident{nat :: nat :: s},
we want the next instruction in the code sequence to accept this shape.

If we regard
$\midt{Instr} : \midt{Shape} \to \midt{Shape} \to \midt{Set}$
as a binary relation on \ident{Shape}, then code is the \emph{transitive reflexive closure}
of \ident{Instr}, which is already included in the Agda standard library as the
module \ident{Data.Star}.

\begin{code}
  -- require import Data.Star
  infixr 5 _<|\_
  data Star {a b : Set} (R : a -> b -> Set) : a -> b -> Set where
    \nil : forall {x} -> Star R x x
    _<|\_ : forall {x y z} -> R x y -> Star R y z -> Star R x z

  Code : Shape -> Shape -> Set
  Code = Star Instr
\end{code}

\noindent The type of instruction sequences is indexed in exactly the same
manner as the type of single instructions: the first index represents the
acceptable shape of stack before execution of the piece of code; the second
index represents the shape of stack after its execution.

Let us conclude this section with an utility function for concatenation of
instruction sequences, which is actually also included in \ident{Data.Star}.

\begin{code}
  infixr 5 _\app\_
  _\app\_ : forall {R x y z} -> Star R x y -> Star R y z -> Star R x z
  \nil \app ys = ys
  (x <| xs) \app ys = x <| xs \app ys
\end{code}

\subsection{Execution}

Now we will describe how the machine executes instructions, that is,
the operational semantics of the low-level language.

At this stage, the state of the machine is fully described by just its stack. This
means that there are no other state variables, registers or any additional
memory.

\subsubsection{Instructions}

Let us describe the effects of single instructions on the state of the machine
(that is, on the stack).

\begin{code}
  execInstr : forall {s t} -> Instr s t -> Stack s -> Stack t
  execInstr (PUSH x) st = x \scons st
  execInstr ADD (x \scons y \scons st) = (x + y) \scons st
\end{code}

\noindent The above function simply says that
\begin{itemize}
  \item the effect of the instruction \ident{PUSH} is pushing the attached
    value onto the stack. This consistently extends the information contained
    in the type of \ident{PUSH x}.\footnote{\ident{Instr s (u $::$ s)} -- for
    some \ident{u} and \ident{s}. This type is interpreted as ,,\ident{PUSH x}
    pushes some value of type \ident{u} onto the stack''.}
    What the type does not say (and \ident{execInstr}
    does) is what value exactly this is.
  \item the effect of the instruction \ident{ADD} pops two \ident{nat}s from
    the top of the stack and pushes their sum back.
\end{itemize}

Note that in this definition, we already reap some benefits of dependently
typed programming.

First, of course, Agda checks types of the terms behind
the scenes and the machinery of types we have designed so far ensures that
in the case for \ident{PUSH x}, pushing the value \ident{x} always yields
a stack of the desired shape.

Second, in the case for \ident{ADD}, the types ensure that there are always two
\ident{nat}s on top of the stack and we can safely pattern-match with the
pattern \ident{x} \scons \ident{y} \scons \ident{st} -- because this match
will always succeed (and no other patterns for the \ident{ADD} case are needed). 

Thus the above definition complies to the type signatures involved (relatively
solid hints of correctness) and it is \emph{total} (esp. no pattern match failures),
while compilers of non-dependently typed languages, like OCaml or Haskell,
would complain about non-exhaustive patterns here --- there is no way to tell them
that, for example, we needn't deal with empty stacks when executing \ident{ADD}.
% TODO: ktory paper tvrdi, ze to je nevyhnutne?

\subsubsection{Code}

Execution of code is then just a left fold over the sequence of instructions,
accepting the initial and yielding the resulting state of the machine.

\begin{code}
  execCode : forall {s t} -> Code s t -> Stack s -> Stack t
  execCode \nil st = st
  execCode (i <| is) st = execCode is (execInstr i st)
\end{code}

\noindent Execution of empty code has no effect on the stack; if the code
contains instructions, then the first instruction is executed and on the
resulting stack, the rest of code is executed.

\subsection{Compiler}

Compiling our simple high-level language for a stack machine is easy. The
central idea is that execution of an expression of some type is equivalent to
pushing its value onto the stack. Literal values are then pushed on the stack
directly; binary-operator expressions first evaluate both operands, effectively
putting their values on the top of the stack, and then execute the appropriate
instruction, determined by the operator. This instruction pops the top
two values from the stack as its operands and pushes the result back.

\begin{code}
  -- Syntactic sugar, promote an Instr to singleton Code
  [[_\;]] : forall {s t} -> Instr s t -> Code s t
  [[i\;]] = i <| \nil

  -- Determine what instruction performs the required calculation
  opInstr : forall {u v w} -> Op u v w -> forall {s} -> Instr (u :: v :: s) (w :: s)
  opInstr Plus = ADD

  -- Turn the expression into code
  compile : forall {u} -> Exp u -> forall {s} -> Code s (u :: s)
  compile (Lit x) = [[ PUSH x ]]
  compile (Bin op l r) = compile r \app compile l \app [[ opInstr op ]]
\end{code}

\noindent Again, behind the scenes, Agda ensures that all types match and the
code compiled by this function will not make the stack machine fail.\footnote{
  To be fair, this is already a property of \ident{Code}, ,,inherited'' by the
  function \ident{compile} via its return type.  However, it does constrain
possible definitions of the function \ident{compile}.} For example, there is no
way to have the function \ident{compile} output code where \ident{ADD} would
not have two \ident{nat}s on the top of the stack.

\subsection{Correctness}

This is the only place in this thesis where we will include the complete proof
of correctness. All proofs are of course contained in the attached Agda source
code.

\subsubsection{Operator lemma}

There are two auxiliary lemmas that we will need to prove our main result. The
first one of them is called \ident{op-correct} and it says that for any binary
operator, the instruction picked by the compiler indeed does what the
denotation of the binary operator says.

To be more specific, for any operator \ident{op} and two values \ident{x} and
\ident{y} of appropriate types, executing \ident{opInstr op} with the two
values on top of the stack results in having the value \ident{denOp op x y}
on the top of the stack afterwards.

\begin{code}
  op\-correct : forall {s u v w} {st : Stack s} {x : el u} {y : el w}
    -> (op : Op u v w)
    -> execInstr (opInstr op) (x \scons y \scons st) == denOp op x y \scons st
  op\-correct Plus = refl
\end{code}

\noindent In the case for \ident{Plus}, Agda substitutes the term \ident{Plus}
in the appropriate places, normalizes the resulting equality (expanding
function definitions etc.) and the proof becomes a trivial observation of
equality of normal forms, which is indicated by \ident{refl}.

\subsubsection{Distributivity lemma}

The other lemma that we will need says that execution of code distributes over
concatenation of code. In other words, executing the code \ident{c $\lhd\!\lhd$
d} has the same effect as first executing \ident{c} and then executing
\ident{d} on the resulting stack.

\begin{code}
  compile\-distr : forall {s t u} {st : Stack s}
    -> (c : Code s t) -> (d : Code t u)
    -> execCode (c \app d) st == execCode d (execCode c st)
\end{code}

\noindent We will proceed by induction on the parameter \ident{c}, which yields
two cases: either \ident{c} is empty or it consists of an instruction and the
rest of code. The first case is trivial by substituting $\varepsilon$ and
comparing the normal forms.

\begin{code}
  compile\-distr \nil d = refl
\end{code}

\noindent For writing the proof for the second case, we will use the wonderful
way supported by the Agda module $\midt{\equiv}$-\ident{Reasoning}\footnote{Actually,
there are also other similar modules, like \ident{$\le$-Reasoning} etc.}, which
lets us write proofs in the equational-reasoning style; just the way we would
do it with pen and paper.

\begin{code}
  compile\-distr (i <| is) d = begin
    execCode (i <| is \app d) st
      ==< refl \>
    execCode (is \app d) (execInstr i st)
      ==< compile\-distr is d \>
    execCode d (execCode is (execInstr i st))
      ==< refl \>
    execCode d (execCode c st)
    \qed
\end{code}

\noindent The proof begins with the word \ident{begin} and continues with the
first line, which is usually exactly the left-hand side of the equality we aim
to prove.  The second line contains the proof that the first line is equal to
the third line and so on -- by alternating terms and equality proofs, we can
gradually rewrite the left-hand term to the right-hand term of the desired
equality.

The first proof is just comparison of normal forms, as indicated by
\ident{refl}.

The second proof uses \ident{compile-distr} recursively as an induction
hypothesis to break the execution of composite code into two stages: first
executing \ident{is}, then executing \ident{d}.

The third proof is just \ident{refl} again and we use it to restructure the
term to the desired final form.

\subsubsection{Alternative proof of distributivity}

% TODO: which version?
Starting from Agda 2.2.4, we can shorten our previous proof considerably
by using the \ident{rewrite} keyword:

\begin{code}
  compile\-distr : forall {s t u} {st : Stack s}
    -> (c : Code s t) -> (d : Code t u)
    -> execCode (c \app d) st == execCode d (execCode c st)
  compile\-distr \nil d = refl
  compile\-distr (i <| is) d rewrite compile\-distr is d (execInstr i st) = refl
\end{code}

\noindent The \ident{rewrite} construct expands to a specific pattern-matching
mechanism behind the scenes, effectively rewriting subterms of the goal using
the provided equality (which is the recursive application of
\ident{compile-distr} here). The resulting goal is then easily solvable by
\ident{refl}.

\subsubsection{Main correctness theorem}

This is the central result of this stage that relates together everything we
have defined so far in a single proof of correctness.

This proof formalizes the idea that we informally mentioned when we started to
write the compiler: executing the compiled code for an expression should be
equivalent to pushing the value of the expression (as given by the denotational
semantics) onto the stack.

\begin{code}
  correctness : forall {u s}
    -> (e : Exp u) (st : Stack s)
    -> execCode (compile e) st == denExp e \scons st
\end{code}

\noindent We will proceed by induction on the expression \ident{e}. The literal
case is trivial and solvable with \ident{refl}.

\begin{code}
  correctness (Lit x) _ = refl
\end{code}

\noindent The binary-operator case is a bit more involved and we will prove it
using equational reasoning, again.

\begin{code}
  correctness (Binop op l r) st = begin
    execCode (compile (Binop op l r)) st
      ==< refl \>
    execCode (compile r \app compile l \app [[ opInstr op ]]) st
      ==< compile\-distr (compile r) _ _ \>
    execCode (compile l \app [[ opInstr op ]]) (execCode (compile r) st)
      ==< compile\-distr (compile l) _ _ \>
    execCode [[ opInstr op ]] (execCode (compile l) (execCode (compile r) st))
      ==< cong (\lam z -> execCode [[ opInstr op ]] (execCode (compile l) z) (correctness r st) \>
    execCode [[ opInstr op ]] (execCode (compile l) (denExp r \scons st))
      ==< cong (\lam z -> execCode [[ opInstr op ]] z) (correctness l st) \>
    execCode [[ opInstr op ]] (denExp l \scons denExp r \scons st)
      ==< refl \>
    execInstr (opInstr op) (denExp l \scons denExp r \scons st)
      ==< op\-correct op \>
    denOp op (denExp l) (denExp r) \scons st
      ==< refl \>
    denExp (Binop op l r) \scons st
    \qed
\end{code}

\noindent The first \ident{refl} is used to expand the definition of compile
for the \ident{Binop} case so that human readers can see what's going on more
easily.

Then we make two appeals to the lemma \ident{compile-distr}. Each usage of this
lemma removes a part of the code sequence (exactly corresponding to an operand
of the binary operator) and transforms it to the effect that this piece of code
has on the stack until only a single instruction is left in the code sequence.
Note that we omit two of three arguments of \ident{compile-distr} in both
applications. This omission improves readability of the proof and Agda can
infer these terms, anyway.

The following two rather cryptic steps use the function \ident{cong} that allows
us to prove equality of two terms, given a proof of equality of their subterms
in a common context.

\begin{code}
  cong : forall {a b : Set} {x y : a}
    -> (f : a -> b)
    -> x == y -> f x == f y
\end{code}

\noindent This function is used with recursive applications of the theorem
\ident{correctness} to both operands of the binary operator. This allows us
to rewrite the subterms in the form \ident{execCode (compile operand) state}
to their equivalents in the form \ident{denExp operand \scons state}. These
two recursive applications are actually inductive hypotheses.

The two steps using \ident{exec-distr} and \ident{correctness} for each operand
actually correspond to ,,accelerated execution'' of these pieces of code -- we
do not execute the instructions; instead, we rely on the induction hypothesis
to simultaneously remove the code corresponding to the operand and push its
denotation onto the stack.

Finally, we use the lemma \ident{op-correct} to show that executing the
leftover instruction is exactly what is left to do to get the desired value
on top of the stack.

\subsection{Remarks}

Totality of Agda functions give us a proof of termination and the above
correctness proof gives us a guarantee that the compiler calculates the correct
code, given the defined semantics.

This is a very strong guarantee and it did not cost us that much -- the code we
have written looks much like the equivalent in any other functional language.
However, we have been maintaining much stronger invariants along the way, being
able to, for example, afford including only \emph{relevant}\footnote{In this
  context, by \emph{relevant} we mean the cases that arise during normal and
  expected operation of the program; for example, as already mentioned, we
needn't specify what to do when the instruction \ident{ADD} gets an
inappropriate number or types of elements on the stack -- just because this
cannot happen \emph{and the compiler knows it}.} pattern cases in a completely
safe way, without triggering compiler warnings.

Implementation-wise, the above (sub-)sections form separate modules in the
accompanying Agda code and these modules define the overall structure of our
development. In the following sections (and chapters), we will develop
the code further by extending and improving particular modules.

\section{Adding exceptions, GMH-style}

Now we are about to add exceptions to our language(s). The first approach is to
examine the ways found in the literature.

Code in this section is modelled after the paper \cite{gmh:exceptions} by
Graham Hutton and Joel Wright. The authors used Haskell in their development.
This approach was later formalized by Tobias Nipkow in \cite{nipkow} in
Isabelle, an interactive theorem prover/proof assistant, in exactly the same
way; the development even uses the same lemmas and their numbering as the
paper.

In contrast with the original paper and Tobias Nipkow's formalization thereof,
our aim is to create a dependently typed program, which means we don't want to
copy the Haskell code as is; instead, we will adapt it for the dependently
typed setting and see how it works out.

\subsection{Changes to the high-level language}

\subsubsection{Expressions}

The first module we need to extend when adding exceptions is the one containing
the definition of expressions of the high-level language. Namely, we need to
add the \ident{Throw} expression and the \ident{Catch} construct.

\begin{code}
  data Op : U -> U -> U -> Set where
    Plus : Op nat nat nat

  data Exp : U -> Set where
    Lit : forall {u} -> el u -> Exp u
    Bin : forall {u v w} -> Op u v w -> Exp u -> Exp v -> Exp w
    Throw : forall {u} -> Exp u
    Catch : forall {u} -> (val : Exp u) -> (hnd : Exp u) -> Exp u
\end{code}

\noindent The type of expressions gets two new constructors.
\begin{itemize}

  \item One of them is \ident{Throw}, which is similar to the literal
    constructor \ident{Lit}, except that no value of the type \ident{el u} is
    needed: a throw-expression can promise to yield a value of any type without
    actually having it.

  \item The other one is \ident{Catch}. This constructor takes two expressions of
    the same type, the regular value and an exception handler, representing a
    catch-block.

\end{itemize}

\subsubsection{Semantics of expressions}

The expression type has just been extended with two new constructors and we
need to formalize what the meaning of the two expression variants actually is.
For that purpose, we need to alter the function \ident{denExp}.

The first change is in the return type of \ident{denExp}: we need a way to
indicate whether the expression evaluates to a value or whether an uncaught
exception occurs. To express that, the function \ident{denExp} will now return
\ident{Maybe (el u)} instead of the more direct \ident{el u}, using the value
\ident{nothing}\footnote{In contrast with Haskell, Agda uses lowercase initials
of the constructors \ident{just} and \ident{nothing}.} to indicate uncaught
exceptions and the value \ident{just x} to indicate that the expression
succesfully evaluates to the value \ident{x}.

The second change is adding pattern cases for the newly added constructors
of \ident{Exp} to the denotation function \ident{denExp}.

\begin{code}
  denExp : forall {u} -> Exp u -> Maybe (el u)
  denExp (Lit x) = just x
  denExp (Bin op l r) with denExp l | denExp r
  \... | just x  | just y  = denOp op x y
  \... | just _  | nothing = nothing
  \... | nothing | just _  = nothing
  \... | nothing | nothing = nothing
  denExp Throw = nothing
  denExp (Catch e h) with denExp e
  \... | just x  = just x
  \... | nothing = denExp h 
\end{code}

\noindent We had to alter the original two cases slightly, most importantly the
binary operator case, where the result now yields a value (i.e.  doesn't throw)
if and only if both subexpressions yield values without throwing.

As hinted above, we added a case for the expression \ident{Throw}: this one
never yields a value and always throws; and also case for catch-expressions: if
no exception gets thrown in the value, the whole catch-expression is equivalent
to the regular value.  Otherwise, it is equivalent to the handler
value.\footnote{Especially, if both values throw exceptions, the
catch-expression propagates the exception thrown in the handler.}
\footnote{Hence, this constructor is similar to the combinator \ident{mplus} in
Haskell, which combines two possibly failing computations in exactly the same
way.}

\subsubsection{Remarks}

To keep things simple, the \ident{Throw} expression does not take a value,
unlike its counterparts in most programming languages. At this stage, we will
worry only about whether an exception has been thrown or not, not about its
particular value.

Also, in most programming languages, exception handlers can inspect the
exceptions being handled and return different values depending on some
attributes of the exception. In our language, it would be pointless to do that
because our exceptions do not carry values. Thus, our exception handlers are
just simple expressions of the same type as the main expression and they have
no means to refer to the exception being handled.

This concludes the definition of our high-level language and the rest of this thesis
will be mostly devoted to how to make it work operationally.

\subsection{Virtual machine}

What about our virtual stack machine and its low-level language of
instructions? What features and instructions do we need to add to make the
machine capable of computing with exceptions?

In \cite{gmh:exceptions}, Hutton and Wright give a description of how this can
be done. Let us extend the machine along the lines drawn by this paper and see
how we can adapt their solution to total functional programming with dependent
types, having the goals from the Introduction (page~\pageref{objectives}) in
mind.

\subsubsection{Stack}

First, Hutton and Wright propose altering the type of stacks because besides
values, now we are going to push exception handlers on the stack, too.

Unlike Hutton and Wright, we also need to care about the type of stack shapes.
This will no longer be a plain
list of types (that is, elements of the universe \ident{U}); instead, we  will
distinguish between \emph{values} and \emph{handlers} pushed on the stack.
\begin{code}
  data Item : Set where
    Val : U -> Item
    Han : U -> Item

  Shape : Set
  Shape = List Item
\end{code}
A value, denoted by \ident{Val u}, is an actual value of the type
denoted by \ident{u}; a handler, denoted by \ident{Han u}, is a piece of code
that, when run, leaves a value of the type \ident{u} on the top of the stack.
This naturally unfolds to the new type of stacks,
\begin{code}
  data Stack : Shape -> Set where
    snil : Stack []
    _\scons\_ : forall {u s} -> el u -> Stack s -> Stack (Val u :: s)
    _\sconsh\_ : forall {u s} -> Code s (Val u :: s) -> Stack s -> Stack (Han u :: s)
\end{code}
where the constructor \ident{\scons\!\!} corresponds to pushing values and the
constructor \ident{\sconsh\!\!} corresponds to pushing handlers on the stack.

Note that we push arbitrarily large strands of code as single items on the stack,
which contradicts one of our design principles -- that the
code must be executable on a simple stack machine (Introduction, page~\pageref{objectives})
-- and we will address this objection in Chapter \ref{chap:compiling2}.

\subsubsection{Instructions and code}

Next, Hutton and Wright introduce three new instructions of the virtual machine:
\ident{MARK}, \ident{UNMARK}, and \ident{THROW}. In our code, this change
reflects in extending the \ident{Instr} type, which must now reside in a
\ident{mutual} block with \ident{Code}:

\begin{code}
  mutual
    data Instr : Shape -> Shape -> Set where
      PUSH : forall {u s} -> el u -> Instr s (Val u :: s)
      ADD : forall s -> Instr (Val nat :: Val nat :: s) (Val nat :: s)
      THROW : forall {u s} -> Instr s (Val u :: s)
      MARK : forall {u s} -> Code s (Val u :: s) -> Instr s (Han u :: s)
      UNMARK : forall {u s} -> Instr (Val u :: Han u :: s) (Val u :: s)

    Code : Shape -> Shape -> Set
    Code = Star Instr
\end{code}

\noindent The definition of code doesn't change: it is still a simple
``index-matching list'' of instructions.

\subsubsection{Machine state, take one}

The topic of machine state is left implicit in the paper by Hutton and Wright,
yet it is one of the most involved parts of this Agda development. The state
certainly contains the stack but -- is it sufficient?

One approach to execution not requiring any additional state would be extending
the stack type with a new constructor;
let us call it \ident{\void\scons\!\!\_}.
\begin{code}
  data Stack : Shape -> Set where
    snil : Stack []
    _\scons\_ : forall {u s} -> el u -> Stack s -> Stack (Val u :: s)
    _\sconsh\_ : forall {u s} -> Code s (Val u :: s) -> Stack s -> Stack (Han u :: s)
    \void\scons\_ : forall {u s} -> Stack s -> Stack (Val u :: s)
\end{code}
The new constructor acts as a placeholder for missing values if an exception is raised.

Note the strong similarity to the \ident{THROW} instruction and the \ident{Throw}
expression. The instruction \ident{THROW} is indexed as an instruction that pushes
a value of any specified type on the stack -- but it actually does not. The 
\ident{Throw} expression is indexed as an expression that yields a value of any specified
type -- but it actually does not. Likewise, the \ident{\void} constructor is typed in
exactly the same way as the constructor that pushes values on the stack -- but it does not.

Thus, it is probably not a surprise that the \ident{THROW} instruction, instead of pushing
a value on the stack, pushes the \ident{\void} placeholder. To be precise, execution would
look the following way.
\begin{code}
  mutual
    execInstr : forall {s t} -> Instr s t -> Stack s -> Stack t
    -- the original two cases
    execInstr (PUSH x) st = x \scons st
    execInstr ADD (x \scons y \scons st) = (x + y) \scons st
    -- new instructions
    execInstr THROW st = \void\scons st
    execInstr (MARK h) st = h \sconsh st
    -- unmark: no exceptions thrown
    execInstr UNMARK (x \scons h \sconsh st) = x \scons st
    -- unmark: an exception thrown, handle it by executing the handler
    execInstr UNMARK (\void\scons h \sconsh st) = execCode h st
    -- miscellaneous exception handling
    execInstr ADD (\void\scons y \scons st) = \void\scons st
    execInstr ADD (x \scons \void\scons st) = \void\scons st
    execInstr ADD (\void\scons \void\scons st) = \void\scons st

    -- execCode is still a left fold over instructions
    execCode : forall {s t} -> Code s t -> Stack s -> Stack t
    execCode \nil st = st
    execCode (i <| is) st = execCode is (execInstr i st)
\end{code}

\noindent However, there are multiple downsides to this solution.

First, and most importantly, it's not really a transition to a different low-level language.
Instead, it is actually evaluation of the function \ident{denExp} with an explicit
stack. The placeholder \ident{\void} corresponds to the outcome \ident{nothing} of
\ident{denExp}, while the regular stack-cons using the constructor \ident{\scons\!\!}
corresponds to the outcome \ident{just x} (for some \ident{x}).

Second, this solution is not elegant in the sense that we need to add exception-handling
cases to every instruction (in our case, only \ident{ADD}). Why should we define how
these instructions handle exceptions when they all must do the same thing: just pass
the exception forward and do nothing else? Exceptions should be handled by a different
mechanism than regular execution.

Third, the Agda termination checker rejects the exception-handling case for
\ident{UNMARK} in the function \ident{execInstr}. Recursion isn't structural
here and we need to convince Agda about termination using a decreasing measure
or another way.

Fourth, it is not how machines actually work. Real exception-handling strategies do jumps,
and they don't push whole code blobs on the stack. Instead, they linearize exception-handling
code with regular code into a single instruction sequence -- and then jump to it if needed.

While this low-level representation does work, it leaves much to be desired. In the following
sections (and chapters), we will address these objections. Let us begin with the first two
of them.

\subsubsection{Machine state, take two}


\section{Termination}





































































