\chapwithtoc{Additional proofs}

\begin{figure}[htp]
\centering
\begin{minipage}{0.45\textwidth}
\begin{prooftree}
\bax{}
\bun{$\alpha:\rho\to\rho \in \Delta, \alpha:\rho\to\rho,\gamma:\rho$}
\alwaysNoLine
\bun{\vdots}
\bun{$A$}
\end{prooftree}
\end{minipage}
%
\begin{minipage}{0.45\textwidth}
\begin{prooftree}
\bax{}
\bun{$\alpha : \rho \in \Delta, \alpha : \rho$}
\alwaysNoLine
\bun{\vdots}
\bun{$B$}
\end{prooftree}
\end{minipage}

\begin{prooftree}
\bax{}
\bright{Ax}\bun{$
\Gamma,x:\rho,f:\rho\to\rho;\Delta,\alpha:\rho\to\rho,\gamma:\rho
	\vdash f : \rho \to \rho
$}
\alwaysNoLine
\bax{$A$}
\bun{$\vdots$}
\alwaysSingleLine
\bright{P}\bbin{$
\Gamma,x:\rho,f:\rho\to\rho;\Delta,\alpha:\rho\to\rho \vdash 
	\lmu{\gamma} [\alpha] f : \rho
$}
\bright{$\to_I$}\bun{$
\Gamma,x:\rho;\Delta,\alpha:\rho\to\rho \vdash 
	(\lam{f:\rho\to\rho} \lmu{\gamma} [\alpha] f) : (\rho \to \rho) \to \rho
$}
\bright{P}\bun{$
\Gamma,x:\rho;\Delta,\alpha:\rho\to\rho \vdash 
	\bigP(\lam{f:\rho\to\rho} \lmu{\gamma} [\alpha] f) : \rho
$}
\bright{$\to_I$}\bun{$
\Gamma;\Delta,\alpha:\rho\to\rho \vdash 
	\big(\lam{x:\rho} \bigP(\lam{f:\rho\to\rho} \lmu{\gamma} [\alpha] f)\big) : \rho\to\rho
$}
\alwaysNoLine
\bax{$B$}
\bun{$\vdots$}
\alwaysSingleLine
\bright{PA}\bbin{$
\Gamma; \Delta \vdash \Big(\lmu{\alpha : \rho \to \rho} [\alpha]
	\big(\lam{x:\rho} \bigP(\lam{f:\rho\to\rho} \lmu{\gamma} [\alpha] f)\big)\Big)
$}
\end{prooftree}
\caption{Derivation of the type of $\bigI$ in $\lambda_\mu$, from \Fref{sec:peirce-law-dynamic}.
The rule PA is the combined Passivate+Activate rule \cite{krebbers11, parigot92}. The rule P
corresponds to the proof that $\bigP$ has the type $((\rho\to\sigma)\to\rho)\to\rho$, as
found in \cite{krebbers11}.}
\label{fig:lambda-mu-identity}
\end{figure}
