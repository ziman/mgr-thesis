%%% Hlavní soubor. Zde se definují základní parametry a odkazuje se na ostatní části. %%%

%% Verze pro jednostranný tisk:
% Okraje: levý 40mm, pravý 25mm, horní a dolní 25mm
% (ale pozor, LaTeX si sám přidává 1in)
%\documentclass[12pt,a4paper]{report}
%\setlength\textwidth{145mm}
%\setlength\textheight{247mm}
%\setlength\oddsidemargin{15mm}
%\setlength\evensidemargin{15mm}
%\setlength\topmargin{0mm}
%\setlength\headsep{0mm}
%\setlength\headheight{0mm}
% \openright zařídí, aby následující text začínal na pravé straně knihy
%\let\openright=\clearpage

%% Pokud tiskneme oboustranně:
\documentclass[12pt,a4paper,twoside,openright]{report}
\setlength\textwidth{145mm}
\setlength\textheight{247mm}
\setlength\oddsidemargin{15mm}
\setlength\evensidemargin{0mm}
\setlength\topmargin{0mm}
\setlength\headsep{0mm}
\setlength\headheight{0mm}
\let\openright=\cleardoublepage

%% Ostatní balíčky
\usepackage{graphicx}
\usepackage{amsfonts, amsmath, amsthm, amssymb}
\usepackage[T1]{fontenc}
\usepackage[utf8]{inputenc}
\usepackage[english]{babel}
\usepackage[all,cmtip]{xy}
\usepackage{textcomp}
\usepackage{subfloat}
\usepackage{graphicx}
\usepackage{lmodern}
\usepackage{microtype}
\usepackage[utf8]{inputenc}
\usepackage{booktabs}
\usepackage{a4wide}
\usepackage{comment}
\usepackage{array}
\usepackage[lofdepth,lotdepth]{subfig}
%\usepackage{natbib}
\usepackage{tgpagella}
%\usepackage{tgcursor}
\usepackage{tikz}
\usetikzlibrary{decorations.pathreplacing,patterns}

%\usepackage{minion}
%\renewcommand{\scdefault}{ssc}
%\newfont{\chapnum}{eurb10 scaled 10000}
%\newcommand{\lowercaps}[1]{\textsc{\MakeLowercase{#1}}}
%\newcommand{\uppercaps}[1]{\textsc{\MakeUppercase{#1}}}

%% Balíček hyperref, kterým jdou vyrábět klikací odkazy v PDF,
%% ale hlavně ho používáme k uložení metadat do PDF (včetně obsahu).
%% POZOR, nezapomeňte vyplnit jméno práce a autora.
\usepackage[colorlinks=true,linkcolor=blue,urlcolor=blue,citecolor=blue,unicode]{hyperref}
\hypersetup{pdftitle=On the semantics of exceptions for high level and low level languages}
\hypersetup{pdfauthor=Matus Tejiscak}

%%% Drobné úpravy stylu

% Tato makra přesvědčují mírně ošklivým trikem LaTeX, aby hlavičky kapitol
% sázel příčetněji a nevynechával nad nimi spoustu místa. Směle ignorujte.
\makeatletter
\def\@makechapterhead#1{
  {\parindent \z@ \raggedright \normalfont
   \Huge\bfseries \thechapter. #1
   \par\nobreak
   \vskip 20\p@
}}
\def\@makeschapterhead#1{
  {\parindent \z@ \raggedright \normalfont
   \Huge\bfseries #1
   \par\nobreak
   \vskip 20\p@
}}
\makeatother

% Toto makro definuje kapitolu, která není očíslovaná, ale je uvedena v obsahu.
\def\chapwithtoc#1{
\chapter*{#1}
\addcontentsline{toc}{chapter}{#1}
}

\begin{document}

% Trochu volnější nastavení dělení slov, než je default.
\lefthyphenmin=2
\righthyphenmin=2

%%% Titulní strana práce

\pagestyle{empty}
\begin{center}

\large

Charles University in Prague

\medskip

Faculty of Mathematics and Physics

\vfill

{\bf\Large MASTER THESIS}

\vfill

\centerline{\mbox{\includegraphics[width=60mm]{figures/logo}}}

\vfill
\vspace{5mm}

{\LARGE Matúš Tejiščák}

\vspace{15mm}

% Název práce přesně podle zadání
{\LARGE\bfseries On the semantics of exceptions for high level and low level languages}

\vfill

% Název katedry nebo ústavu, kde byla práce oficiálně zadána
% (dle Organizační struktury MFF UK)
Department of Theoretical Computer Science and Mathematical Logic

\vfill

\begin{tabular}{rl}

Supervisor of the master thesis: & Wouter Swierstra \\
\noalign{\vspace{2mm}}
Study programme: & Computer science \\
\noalign{\vspace{2mm}}
Specialization: & Theoretical computer science \\
\end{tabular}

\vfill

% Zde doplňte rok
Prague 2012

\end{center}

\newpage

%%% Následuje vevázaný list -- kopie podepsaného "Zadání diplomové práce".
%%% Toto zadání NENÍ součástí elektronické verze práce, nescanovat.

%%% Na tomto místě mohou být napsána případná poděkování (vedoucímu práce,
%%% konzultantovi, tomu, kdo zapůjčil software, literaturu apod.)

\openright

\noindent
Dedication. % TODO

\newpage

%%% Strana s čestným prohlášením k diplomové práci

\vglue 0pt plus 1fill

\noindent
I declare that I carried out this master thesis independently, and only with the cited
sources, literature and other professional sources.

\medskip\noindent
I understand that my work relates to the rights and obligations under the Act No.
121/2000 Coll., the Copyright Act, as amended, in particular the fact that the Charles
University in Prague has the right to conclude a license agreement on the use of this
work as a school work pursuant to Section 60 paragraph 1 of the Copyright Act.

\vspace{10mm}

\hbox{\hbox to 0.5\hsize{%
In Prague, date .....................
\hss}\hbox to 0.5\hsize{%
signature of the author
\hss}}

\vspace{20mm}
\newpage

%%% Povinná informační strana diplomové práce

\vbox to 0.5\vsize{
\setlength\parindent{0mm}
\setlength\parskip{5mm}

Název práce:
On the semantics of exceptions for high level and low level languages
% přesně dle zadání

Autor:
Matúš Tejiščák

Katedra:  % Případně Ústav:
Katedra teoretické informatiky a matematické logiky
% dle Organizační struktury MFF UK

Vedoucí diplomové práce:
Wouter Swierstra PhD, Software Technology Group of Utrecht University
% dle Organizační struktury MFF UK, případně plný název pracoviště mimo MFF UK

Abstrakt:
TODO
% abstrakt v rozsahu 80-200 slov; nejedná se však o opis zadání diplomové práce

Klíčová slova:
závisle typované programování, výjimky, kompilátor, korektnost
% 3 až 5 klíčových slov

\vss}\nobreak\vbox to 0.49\vsize{
\setlength\parindent{0mm}
\setlength\parskip{5mm}

Title:
On the semantics of exceptions for high level and low level languages
% přesný překlad názvu práce v angličtině

Author:
Matúš Tejiščák

Department:
Department of Theoretical Computer Science and Mathematical Logic
% dle Organizační struktury MFF UK v angličtině

Supervisor:
Wouter Swierstra PhD, Software Technology Group of Utrecht University
% dle Organizační struktury MFF UK, případně plný název pracoviště
% mimo MFF UK v angličtině

Abstract:
TODO
% abstrakt v rozsahu 80-200 slov v angličtině; nejedná se však o překlad
% zadání diplomové práce

Keywords:
dependently-typed programming, exceptions, compiler, correctness
% 3 až 5 klíčových slov v angličtině

\vss}

\newpage

%%% Strana s automaticky generovaným obsahem diplomové práce. U matematických
%%% prací je přípustné, aby seznam tabulek a zkratek, existují-li, byl umístěn
%%% na začátku práce, místo na jejím konci.

\openright
\pagestyle{plain}
\setcounter{page}{1}
\tableofcontents

%%% Jednotlivé kapitoly práce jsou pro přehlednost uloženy v samostatných souborech
\chapter*{Introduction}
\addcontentsline{toc}{chapter}{Introduction}

% SPJ says:
% 1. Describe the problem
% 2. State your contributions
% ...and that's all.

This thesis deals with the semantics of exceptions in programming languages and how the semantics
is preserved by the compilation process.

To verify a~compiler that compiles code from a~high-level language (such as Haskell)
to a~low-level language (such as the x86 assembler), we need (besides other things):
\begin{itemize}
	\item to \emph{state semantics} and properties of the \emph{high-level} language;
	\item to \emph{state semantics} for \emph{low-level} code that results from the compilation process;
	\item to \emph{prove} that compilation preserves this semantics;
	\item to \emph{formalize} the above three points in a~way that allows for mechanical verification.
\end{itemize}
Clearly, without these ingredients, there is no way to define what correctness of a~compiler actually
means (let alone prove it).

In this thesis, we pursue the above four goals, focused on a simple language with exceptions.
Besides these core objectives, we want our solution to have other properties, all related
to being useful in practical compiler development:

\begin{itemize}\label{objectives}

	\item the specifications should be \emph{runnable}; that is, the result should be a~program
		that can be run, that can provide compiled code for given input code fragments, and
		that can execute the compiled code, producing actual results;
		
	\item the program should be \emph{readable}. Even though the program may be verified,
		it should still convey the mechanism of execution clearly without proof clutter getting
		in the way of comprehension;
		
	\item the compiled code should be \emph{executable by a~simple machine} (a~stack machine,
		for instance), with fully explicit state and no fancy high-level features such as
		continuations, arbitrarily-sized instructions or implicit stacks;
		
	\item there should be an~obvious and \emph{straightforward way to extract} the executable core
		sans proofs into other (mainstream) languages (either manually or automatically)
		for practical use.
		
\end{itemize}

\noindent The contributions of this thesis are:
\begin{itemize}
	\item We formalize the discussed semantics for a~simple language of expressions
		featuring binary operators\footnote{We actually implement only addition but adding new
		operators is trivial.} and exceptions, using the language as our high-level language
		(Section \ref{sec:expression-semantics});
	\item we explore approaches to the design of execution of low-level code on a~virtual
		machine, first examining an approach found in the
		literature, then presenting a~modified one that is easier to implement in a~dependent
		setting by gradually improving on a naive solution and explaining the choices
		(Sections \ref{sec:own-execution-gmh} to \ref{sec:linearized-code});
	\item we formalize the semantics of a~language of instructions for a~simple stack machine,
		which is our target low-level language, using the latter approach
		(Section \ref{sec:lin-instr-semantics});
	\item we define a~compiler generating instruction sequences from expressions in the high-level
		language (Section \ref{sec:lin-compile});
	\item we prove that the semantics is preserved in code generated by this compiler
		(Section \ref{sec:lin-correctness}).
\end{itemize}

\noindent We use the general-purpose dependently typed pure functional language Agda in
our development.
Readers of this thesis are assumed to have knowledge of functional programming, but
not necessarily dependently typed programming. A~brief introduction to dependent types and
related concepts for the purposes of this thesis will be given in Chapter \ref{chap:dependent-types}.

\chapter{Exceptions in programming languages}

\section{Purpose}

Blah blah blah blah. Lots of fun, as described in \cite{swierstra:thesis}.

Some sample code:
\begin{code}
data Nat : forall (a : Set) (x : a) -> a -> a -> Set where
  zero : forall {x} -> Nat x x  -- zero constructor
  suc : forall {x} -> Nat x x -> Nat x x  -- successor
\end{code}

\section{History}

\section{Theoretical consequences}


\chapter{Dependently-typed programming}

\section{Dependent types}

\section{The Curry-Howard correspondence}

\section{Total functional programming}

\subsection{Structural recursion and termination}

\section{Formal verification}

\section{The Agda proof assistant}

\chapter{Compiling exceptions}

The basic approach to compiling exceptions was shown in \cite{gmh:exceptions}.
This article provides excellent insight and inspiration how such code can be written
in a language such as Haskell.
However, peculiarities of programming with dependent types will make us
diverge a bit, in details at first, in parts of the design later.

\section{A simple exceptionless language}

All high-level languages in this thesis will be languages of simple, typed
expressions.

Our first language will feature only natural numbers and addition, not even
exceptions. We will implement it to build the auxiliary ecosystem of the
compiler and the skeleton of our development.

\subsection{Type universe}

The first thing we will introduce is the type universe representing the set of
types of expressions of the high-level language. We will index our types with
values of this type (most notably the type \ident{Exp u} of expressions, to
indicate what the type of the expression is), and while we could use Agda types
directly for indexing, with an explicit universe, we get decidable equality of
types and all datatypes conveniently in \ident{Set}. We could parametrize our
modules with the universe but let us just define a fixed one for the sake of
simplicity.

\begin{code}
  infixr 5 _=>_
  data U : Set where
    nat : U
    _=>_ : U -> U -> U
\end{code}

\noindent We will also need the interpretation function \ident{el} that maps
types of our simple language to Agda types so that we can use Agda values in
the modelled language.

\begin{code}
  el : U -> Set
  el nat = Nat
  el (s => t) = el s -> el t
\end{code}

\noindent And, as promised, equality is decidable on the elements of the
universe.

\begin{code}
  -- open import Relation.Nullary using (Dec)
  data Dec (p : Set) : Set where
    yes : p -> Dec p
    no  : ~ p -> Dec p

  -- open import Relation.Binary.PropositionalEquality using (_==_)
  data _==_ {a : Set} (x : a) : a -> Set where
    refl : x == x

  eqDecU : (u v : U) -> Dec (u == v)
  -- body omitted
\end{code}

\noindent We omit the body of \ident{eqDecU} because it's just an ordinary
uninteresting case analysis.

\subsection{Expressions}

The core of the language we are going to model consists of its expressions, of
course. For now, we will support nothing more than (numeric) literals and addition.
However, for further extensibility, we separate the type of binary operators.

The type of binary operators is indexed by the types of the two values that
the operator accepts as arguments; the third index represents the type of
the result of application of the operator on the two values.

\begin{code}
  data Op : U -> U -> U -> Set where
    Plus : Op nat nat nat
\end{code}

\noindent Now we can define the expressions: literals and binary operators.
Note that we index the datatype with elements of the universe \ident{U} so that
the Agda type of an expression also incorporates the type of the expression in
the modelled language.

\begin{code}
  data Exp : U -> Set where
    -- Literals
    Lit : forall {u} -> el u -> Exp u
    -- Binary operators
    Bin : forall {u v w} -> Bin u v w -> Exp u -> Exp v -> Exp w
\end{code}

These two short bits of code make up for the complete definition of the expressions
of our first high-level language.

\subsection{Virtual machine}

We will use a very simple stack machine to run the compiled code.

\subsubsection{Stack}

The stack of
the machine is just a cons-list of values, indexed by types (elements of
\ident{U}) of the values pushed on the stack.  This means that just by looking
at the type of the stack, we can tell how many elements it contains and what
types they have.
\begin{code}
  -- open import Data.List
  infixr 5 _::_
  data List (a : Set) : Set where
    [] : List a
    _::_ : a -> List a -> List a

  Shape : Set
  Shape = List U

  infixr 5 _\scons_
  data Stack : Shape -> Set where
    snil : Stack []
    _\scons_ : forall {u s} -> el u -> Stack s -> Stack (u :: s)
\end{code}
\noindent The literal \ident{snil} represents the empty stack; new values are
pushed onto it using the infix constructor \ident{\_\scons\_}.

\subsubsection{Instructions}

At this stage, the machine supports only two instructions: \ident{PUSH}
and \ident{ADD}.
\begin{code}
  data Instr : Shape -> Shape -> Set where
    PUSH : forall {u s} -> el u -> Instr s (u :: s)
    ADD : forall s -> Instr (nat :: nat :: s) (nat :: s)
\end{code}
The type of instructions is indexed by their action on stack. The first shape
argument is the required stack shape so that the instruction can be executed;
the second shape argument is the resulting shape of the stack after the
instruction has been executed.

For example, the instruction \ident{PUSH} takes any value of the type \ident{el u}
and pushes it onto a stack having any shape \ident{s}, creating a new
stack of the shape \ident{u :: s}.

The instruction \ident{ADD} represents popping two natural numbers from the
stack of any shape with two \ident{nat}s on top of it, (hence
\ident{nat :: nat :: s})
and subsequently pushing their sum onto it, resulting in the shape
\ident{nat :: s}.

\subsubsection{Code}

Finally, code for the stack machine is a sequence of instructions, where
type indices of subsequent instructions match. For example, if one instruction
in the sequence produces a stack of the shape \ident{nat :: nat :: s},
we want the next instruction in the code sequence to accept this shape.

If we regard
$\midt{Instr} : \midt{Shape} \to \midt{Shape} \to \midt{Set}$
as a binary relation on \ident{Shape}, then code is the \emph{transitive reflexive closure}
of \ident{Instr}, which is already included in the Agda standard library as the
module \ident{Data.Star}.

\begin{code}
  -- require import Data.Star
  infixr 5 _<|_
  data Star {a b : Set} (R : a -> b -> Set) : a -> b -> Set where
    \nil : forall {x} -> Star R x x
    _<|_ : forall {x y z} -> R x y -> Star R y z -> Star R x z

  Code : Shape -> Shape -> Set
  Code = Star Instr
\end{code}

\noindent The type of instruction sequences is indexed in exactly the same
manner as the type of single instructions: the first index represents the
acceptable shape of stack before execution of the piece of code; the second
index represents the shape of stack after its execution.

Let us conclude this section with an utility function for concatenation of
instruction sequences, which is actually also included in \ident{Data.Star}.

\begin{code}
  infixr 5 _\app_
  _\app_ : forall {R x y z} -> Star R x y -> Star R y z -> Star R x z
  _\app_ \nil ys = ys
  _\app_ (x <| xs) ys = x <| xs \app ys
\end{code}

\subsection{Compiler}

\subsection{Execution}

\subsection{Correctness}

\section{Adding exceptions, GMH-style}

\section{Termination}

\chapter{Verified and total exception compiler}

\section{Adding exceptions, take two}

\section{Linearization}

\chapter{Discussion}
%\addcontentsline{toc}{chapter}{Conclusion}

\todo{What we have achieved}

\section{Related work}

\section{Further work}

Some further work.

\todo{Forks}

\todo{Measure + GMH approach}

\todo{Hyperdependent code}

\todo{Lambdas}

\todo{Statically typed exceptions}

\todo{Peirce (statically \emph{bound} exceptions)}

\todo{Well-founded recursion, completely machine-like code}

\todo{Scrap types, use stack sizes}

\todo{How should the attached code be structured and referenced from the thesis? Insert references in all appropriate places.}

\todo{Change ! to $\times$ and vice versa.}

\todo{Snoc-code is not tail-recursive}

\todo{$12$ clauses per binop in correctness lemma: factorize?}

\todo{Add non-breaking spaces}

\todo{Add real-world counterparts, like saying "here the machine would use a label and then just jump to it".}

\todo{Goal: execution should be tail-recursive}

\todo{Finally, we converge to the solution by GMH, just from a different direction.}

\todo{Spell-check}

\todo{List prerequisites? haskell familiarity? in the introduction?}

\todo{add extensibility as a goal?}

\todo{Search for \texttt{\textbackslash def \textbackslash @textbottom\{\textbackslash vskip
\textbackslash z@ \textbackslash@plus 14pt\}} in the headers and decrease those 14 pts to
something sane when everything is done.}

\todo{Operator precedence is $0$--$10$, not $0$--$100$.}

\todo{Check quote ``consistency'' vs. ,,consistency''.}

\todo{Insert references to external Agda tutorials.}

\todo{Consider wording: data type vs. type family. Simple or precise?}

\todo{Proper italics in bibliography. Proper ISO bibliography.}

\todo{Proper typography, capitalization, grammar in bibliography.}

\todo{When will the new bridge be built? (Search for ``can be''.)}

\todo{Discuss the compiler written in Epigram (paper received by mail).}

\todo{Mention CompCert and the Leroy's paper.}

\section{Conclusions}


%%% Seznam použité literatury
\include{bibliography}

%%% Tabulky v diplomové práci, existují-li.
\chapwithtoc{List of Tables}

%%% Použité zkratky v diplomové práci, existují-li, včetně jejich vysvětlení.
\chapwithtoc{List of Abbreviations}

%%% Přílohy k diplomové práci, existují-li (různé dodatky jako výpisy programů,
%%% diagramy apod.). Každá příloha musí být alespoň jednou odkazována z vlastního
%%% textu práce. Přílohy se číslují.
\chapwithtoc{Attachments}

\openright
\end{document}
