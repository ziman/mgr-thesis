\frontmatter

% Trochu volnější nastavení dělení slov, než je default.
\lefthyphenmin=2
\righthyphenmin=2

%%% Titulní strana práce

\pagestyle{empty}
\begin{center}

\large

Charles University in Prague

\medskip

Faculty of Mathematics and Physics

\vfill

{\bf\Large MASTER THESIS}

\vfill

\centerline{\mbox{\includegraphics[width=60mm]{figures/logo}}}

\vfill
\vspace{5mm}

{\LARGE Matúš Tejiščák}

\vspace{15mm}

% Název práce přesně podle zadání
{\LARGE\bfseries On the semantics of exceptions for high level and low level languages}

\vfill

% Název katedry nebo ústavu, kde byla práce oficiálně zadána
% (dle Organizační struktury MFF UK)
Department of Theoretical Computer Science and Mathematical Logic

\vfill

\begin{tabular}{rl}

Supervisor of the master thesis: & Wouter Swierstra PhD\\
\noalign{\vspace{2mm}}
Study programme: & Computer science \\
\noalign{\vspace{2mm}}
Specialization: & Theoretical computer science \\
\end{tabular}

\vfill

% Zde doplňte rok
Prague 2012

\end{center}

\newpage

%%% Následuje vevázaný list -- kopie podepsaného "Zadání diplomové práce".
%%% Toto zadání NENÍ součástí elektronické verze práce, nescanovat.
\todo{Voviazat zadanie}

%%% Na tomto místě mohou být napsána případná poděkování (vedoucímu práce,
%%% konzultantovi, tomu, kdo zapůjčil software, literaturu apod.)
\todo{Thanks: Wouter, Flu, Freek+Herman+team}

\openright

\noindent
I would like to thank to my dear Flu for giving me the key initial push to leave for
Nijmegen -- thus triggering a sequence of exciting events leading to this work --
and for her support all along the way.

Not the smallest bit less, I would like to thank Dr. Wouter Swierstra for his excellent
guidance, from the point of helping a confused student pick a topic, through consultations
conveying experience and insight while leaving decisions up to me, to the final reviews few days
before submission of the thesis.
\newpage

%%% Strana s čestným prohlášením k diplomové práci

\vglue 0pt plus 1fill

\noindent
I declare that I carried out this master thesis independently, and only with the cited
sources, literature and other professional sources.

\medskip\noindent
I understand that my work relates to the rights and obligations under the Act No.
121/2000 Coll., the Copyright Act, as amended, in particular the fact that the Charles
University in Prague has the right to conclude a license agreement on the use of this
work as a school work pursuant to Section 60 paragraph 1 of the Copyright Act.

\vspace{10mm}

\hbox{\hbox to 0.5\hsize{%
In Prague, date .....................
\hss}\hbox to 0.5\hsize{%
signature of the author
\hss}}

\vspace{20mm}
\newpage

%%% Povinná informační strana diplomové práce

\vbox to 0.5\vsize{
\setlength\parindent{0mm}
\setlength\parskip{5mm}

Název práce:
On the semantics of exceptions for high level and low level languages
% přesně dle zadání

Autor:
Matúš Tejiščák

Katedra:  % Případně Ústav:
Katedra teoretické informatiky a matematické logiky
% dle Organizační struktury MFF UK

Vedoucí diplomové práce:
Wouter Swierstra PhD, Software Technology Group of Utrecht University
% dle Organizační struktury MFF UK, případně plný název pracoviště mimo MFF UK

Abstrakt:
V práci se zabýváme korektností kompilátoru jazyka s výjimkami. Předkládáme formální sémantiku;
jak denotační sémantiku vysokoúrovňového jazyka, tak operační sémantiku jazyka instrukcí pro
zásobníkový stroj. Studujeme metodu odvíjení zásobníku a poté,
iterativně ve více krocích, předkládáme modifikovanou metodu. Tato je strukturálně rekurzivní
a tudíž vhodná pro implementaci v totálních závisle typovaných jazycích. Nakonec předkládáme
implementaci kompilátoru v závisle typovaném jazyce Agda,
spolu se strojově ověřitelným důkazem, že předložená implementace kompilátoru při překladu
zachovává sémantiku programu. 

Klíčová slova:
závisle typované programování, výjimky, kompilátor, korektnost
% 3 až 5 klíčových slov

\vss}\nobreak\vbox to 0.49\vsize{
\setlength\parindent{0mm}
\setlength\parskip{5mm}

Title:
On the semantics of exceptions for high level and low level languages
% přesný překlad názvu práce v angličtině

Author:
Matúš Tejiščák

Department:
Department of Theoretical Computer Science and Mathematical Logic
% dle Organizační struktury MFF UK v angličtině

Supervisor:
Wouter Swierstra PhD, Software Technology Group of Utrecht University
% dle Organizační struktury MFF UK, případně plný název pracoviště
% mimo MFF UK v angličtině

Abstract:
The thesis deals with correctness of a compiler of a simple language featuring exceptions.
We present formal semantics, both denotational semantics of a~high-level language and
operational semantics of a low-level language for a simple stack machine.
We study the method of stack unwinding and then iteratively, improving upon a naive solution,
we present a different method
that is structurally recursive and thus suitable for implementation in total dependently
typed languages.
Finally, we provide
an implementation of the compiler in the dependently typed functional programming language
Agda, along with a mechanically verifiable proof of adherence of the implementation to the
semantics.
% abstrakt v rozsahu 80-200 slov v angličtině; nejedná se však o překlad
% zadání diplomové práce

Keywords:
dependently-typed programming, exceptions, compiler, correctness
% 3 až 5 klíčových slov v angličtině

\vss}
