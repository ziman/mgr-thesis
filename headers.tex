%% Verze pro jednostranný tisk:
% Okraje: levý 40mm, pravý 25mm, horní a dolní 25mm
% (ale pozor, LaTeX si sám přidává 1in)
%\documentclass[12pt,a4paper]{report}
%\setlength\textwidth{145mm}
%\setlength\textheight{247mm}
%\setlength\oddsidemargin{15mm}
%\setlength\evensidemargin{15mm}
%\setlength\topmargin{0mm}
%\setlength\headsep{0mm}
%\setlength\headheight{0mm}
% \openright zařídí, aby následující text začínal na pravé straně knihy
%\let\openright=\clearpage

%% Pokud tiskneme oboustranně:
\documentclass[12pt,a4paper,twoside,openright]{book}
\setlength\textwidth{145mm}
%\setlength\textheight{247mm}
\setlength\oddsidemargin{15mm}
\setlength\evensidemargin{0mm}
%\setlength\topmargin{0mm}
%\setlength\headsep{0mm}
%\setlength\headheight{0mm}
\let\openright=\cleardoublepage

%% Ostatní balíčky
\usepackage{graphicx}
\usepackage{amsfonts, amsmath, amsthm, amssymb}
\usepackage[T1]{fontenc}
\usepackage[utf8]{inputenc}
\usepackage[english]{babel}
\usepackage[all,cmtip]{xy}
\usepackage{textcomp}
\usepackage{subfloat}
\usepackage{graphicx}
\usepackage{lmodern}
\usepackage{microtype}
\usepackage[utf8]{inputenc}
\usepackage{booktabs}
\usepackage{a4wide}
\usepackage{comment}
\usepackage{array}
\usepackage[lofdepth,lotdepth]{subfig}
%\usepackage{natbib}
\usepackage{tgpagella}
%\usepackage{tgcursor}
\usepackage{tikz}
\usetikzlibrary{decorations.pathreplacing,patterns}

\newcommand{\scons}{:-:}

\usepackage{listings}
\lstloadlanguages{Haskell}
\lstnewenvironment{code}
    {\lstset{}%
      \csname lst@SetFirstLabel\endcsname}
    {\csname lst@SaveFirstLabel\endcsname}
    \lstset{
      comment=[l]--,
      basicstyle=\small\sffamily,
      keywordstyle=\bf\sffamily,
      commentstyle=\color{gray}\it\rmfamily,
      columns=fullflexible,
      basewidth={0.5em,0.45em},
%      xleftmargin=1em,
      morekeywords={data,where,open,import,let,with,module,using},
      literate=
      	{+}{{$+\:$}}1
      	{/}{{$/$}}1
      	{*}{{$*$}}1
      	{=}{{$=\;$}}1
        {>}{{$>$}}1
        {<}{{$<$}}1
        {\\lam}{{$\lambda$}}1
        {\\\\}{{\char`\\\char`\\}}1
       	{->}{{$\rightarrow\;$}}2
       	{>=}{{$\geq$}}2
       	{<-}{{$\leftarrow$}}2
        {<=}{{$\leq$}}2
        {=>}{{$\Rightarrow\;$}}2 
        {\ .}{{$\circ$}}2
        {\ .\ }{{$\circ$}}2
        {>>}{{>>}}2
        {>>=}{{>>=}}2
        {|}{{$\mid$}}1
        {Nat}{{$\mathbb{N}\:$}}1
        {forall}{{$\forall\:$}}1
        {\\scons}{{\scons}}1
  		  {\\nil}{{$\varepsilon\:$}}1
	  	  {==}{{$\equiv\:$}}1
		    {<|}{{$\lhd\;$}}1
        {~}{{$\neg\:$}}1
    }

%\usepackage{minion}
%\renewcommand{\scdefault}{ssc}
%\newfont{\chapnum}{eurb10 scaled 10000}
%\newcommand{\lowercaps}[1]{\textsc{\MakeLowercase{#1}}}
%\newcommand{\uppercaps}[1]{\textsc{\MakeUppercase{#1}}}

%% Balíček hyperref, kterým jdou vyrábět klikací odkazy v PDF,
%% ale hlavně ho používáme k uložení metadat do PDF (včetně obsahu).
%% POZOR, nezapomeňte vyplnit jméno práce a autora.
\usepackage{color}
\definecolor{link}{rgb}{0,0,0.5}
\usepackage[
	colorlinks=true,
	linkcolor=link,
	urlcolor=link,
	citecolor=link,
	unicode
]{hyperref}
\hypersetup{pdftitle=On the semantics of exceptions for high level and low level languages}
\hypersetup{pdfauthor=Matus Tejiscak}

%%% Drobné úpravy stylu

% Tato makra přesvědčují mírně ošklivým trikem LaTeX, aby hlavičky kapitol
% sázel příčetněji a nevynechával nad nimi spoustu místa. Směle ignorujte.
%\makeatletter
%\def\@makechapterhead#1{
%  {\parindent \z@ \raggedright \normalfont
%   \Huge\bfseries \thechapter. #1
%   \par\nobreak
%   \vskip 20\p@
%}}
%\def\@makeschapterhead#1{
%  {\parindent \z@ \raggedright \normalfont
%   \Huge\bfseries #1
%   \par\nobreak
%   \vskip 20\p@
%}}
%\makeatother

% Toto makro definuje kapitolu, která není očíslovaná, ale je uvedena v obsahu.
\def\chapwithtoc#1{
\chapter*{#1}
\addcontentsline{toc}{chapter}{#1}
}

\newcommand{\ident}[1]{\textsf{#1}}
\newcommand{\midt}[1]{\mathsf{#1}}
%\newcommand{\to}{\rightarrow}

